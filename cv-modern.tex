%% start of file `template.tex'.
%% Copyright 2006-2010 Xavier Danaux (xdanaux@gmail.com).
%
% This work may be distributed and/or modified under the
% conditions of the LaTeX Project Public License version 1.3c,
% available at http://www.latex-project.org/lppl/.


\documentclass[11pt,letterpaper]{moderncv}

% moderncv themes
\moderncvtheme[blue]{casual}                 % optional argument are 'blue' (default), 'orange', 'red', 'green', 'grey' and 'roman' (for roman fonts, instead of sans serif fonts)
%\moderncvtheme[green]{classic}                % idem

% character encoding
\usepackage[utf8]{inputenc}                   % replace by the encoding you are using

% adjust the page margins
\usepackage[scale=0.8]{geometry}
\setlength{\hintscolumnwidth}{3.5cm}						% if you want to change the width of the column with the dates
%\AtBeginDocument{\setlength{\maketitlenamewidth}{6cm}}  % only for the classic theme, if you want to change the width of your name placeholder (to leave more space for your address details
%\AtBeginDocument{\recomputelengths}                     % required when changes are made to page layout lengths

% How to change the bibliography heading title
\renewcommand\refname{Artículos y publicaciones}

% personal data
\firstname{Willington}
\familyname{Vega Cardona}
\title{Willington Vega Cardona}
\address{Calle 39A \# 110-240 INT 301}{Medellín, Colombia}
\mobile{316 632 61 98}
\phone{(+57) 4 496 07 67}
%\fax{fax (optional)}
\email{wvega@wvega.com}
\homepage{http://wvega.com/}
%\extrainfo{additional information (optional)}
\photo[64pt]{empty}                         % '64pt' is the height the picture must be resized to and 'picture' is the name of the picture file;
%\quote{Septiembre, 2010}

% to show numerical labels in the bibliography; only useful if you make citations in your resume
\makeatletter
\renewcommand*{\bibliographyitemlabel}{\@biblabel{\arabic{enumiv}}}
\makeatother

% bibliography with mutiple entries
%\usepackage{multibib}
%\newcites{book,misc}{{Books},{Others}}

%\nopagenumbers{}                             % uncomment to suppress automatic page numbering for CVs longer than one page
%----------------------------------------------------------------------------------
%            content
%----------------------------------------------------------------------------------
\begin{document}

\maketitle

\section{Datos personales}
\cvline{Cédula de Ciudadanía}{1.128.417.376}
\cvline{Fecha de Nacimiento}{Agosto 24, 1988}
\cvline{Lugar de Nacimiento}{Granada, Antioquia}
\cvline{Dirección}{Calle 39A \# 110 - 240 INT 301 $-$ Medellín, Colombia}
\cvline{Teléfono}{(+57) 4 496 07 67}
\cvline{Celular}{(+57) 316 632 61 98}
\cvline{Email}{\href{mailto:wvega@wvega.com}{wvega@wvega.com}}
\cvline{Página Web}{\href{http://wvega.com/}{http://wvega.com/}}

\section{Formación académica}
% arguments 3 to 6 can be left empty
%\cventry{year--year}{Degree}{Institution}{City}{\textit{Grade}}{Description}
\cventry{2005--2010}{Ingeniero de Sistemas e Informática}{Universidad Nacional de Colombia}{Medellín}{}{
Trabajo de grado:\\
\emph{"Predicción de la demanda mensual de electricidad usando máquinas de vectores de soporte"}.\\
Fecha esperada de graduación: Diciembre 2010}

\section{Nombramientos / Honores}

\cvline{ECAES 2009--02}{Ubicado entre los mejores 10 puntajes para Ingeniería de Sistemas en Colombia. Mejor puntaje para Ingeniería de Sistemas e Informática en la Universidad Nacional de Colombia, Sede Medellín.}

\cvline{2009--01}{Exención pago de matrícula. Universidad Nacional de Colombia, Sede Medellín.}

\newpage

% Publications from a BibTeX file without multibib\renewcommand*{\bibliographyitemlabel}{\@biblabel{\arabic{enumiv}}}% for BibTeX numerical labels
\nocite{*}
\bibliographystyle{ieeetr}
\bibliography{publications}       % 'publications' is the name of a BibTeX file

\section{Eventos y conferencias}

\cventry{2007}{Festival Latinoamericano de Instalación de Software Libre}{Medellín (Colombia)}{}{}{
Comité Organizador\\
\href{http://unalix.org/Eventos/FLISOL2007}{http://unalix.org/Eventos/FLISOL2007}}

\cventry{2008}{3CCC - Tercer Congreso Colombiano de Computación}{Univerisidad EAFIT}{Medellín (Colombia)}{20 horas}{Asistente}

\section{Proyectos}

\cventry{2006--2008}{Construcción de un editor genérico para la elaboración de modelos gráfico simbólicos}{Auxiliar de investigación}{}{(C++)}{
Ph.D Fernando Arango Isaza (Director del proyecto).\\
20 horas semanales.}

\smallskip

\cventry{2006--2010}{UNALIX}{}{}{}{
Grupo de software libre de la Universidad Nacional de Colombia, Sede Medellín.\\
\href{http://www.unalix.org/}{http://www.unalix.org/}}

\smallskip

\cventry{2007}{QtOctave}{Desarrollador asociado}{}{(C++, Qt)}{QtOctave es una interfaz gráfica de usuario para el lenguaje de alto nivel Octave, diseñado principalmente para realizar cálculos numéricos de forma similar a Matlab. QtOctave simplifica el uso de Octave para los usuarios poco familiarizados con la versión original de este software, basada en comandos.\\
\href{http://forja.rediris.es/projects/csl-qtoctave/}{https://forja.rediris.es/projects/csl-qtoctave/}}

\smallskip

\cventry{2010}{Desarrollo de un generador de preguntas para el curso programación lógica y funcional}{Desarrollador}{}{(PHP)}{
PETA dirigido por Ph.D Fernando Arango Isaza.\\
12 horas semanales.}

\smallskip

\newpage

\section{Habilidades}

\subsection{Habilidades informáticas}

\cvcomputer{\textbf{Lenguajes de programación}}{Python, C, C++, PHP, Java, JavaScript, Scheme, MAUDE, R}{\textbf{Servicios}}{Apache, Tomcat,\\ LDAP, DNS, DHCP}
\cvcomputer{\textbf{Bases de datos}}{MySQL, Oracle, SQLite}{\textbf{Control de versiones}}{Subversion, Git}
\cvcomputer{\textbf{Entornos de programación}}{NetBeans, Eclipse}{\textbf{Diseño gráfico}}{Inkscape, GIMP,\\ Photoshop}
\cvcomputer{\textbf{Sistemas operativos}}{GNU Linux (Fedora, Ubuntu,\\ CentOS), Microsoft Windows\\ (XP, Vista, 7)}{\textbf{Herramientas ofimáticas}}{OpenOffice, \\Microsoft Office 2003, \\Microsoft Office 2007}

\section{Idiomas}

\cvlanguage{Español}{Nativo}{}
\cvlanguage{Inglés}{Nivel Medio}{\textbf{Lectura:} Fluida, \textbf{Escritura:} Fluida, \textbf{Habla:} Nivel Medio}

\section{Intereses}

\cvcomputer{\textbf{Académicos}}{Máquinas de Vectores de Soporte(SVM), Aprendizaje de Máquina, Sistemas Multi-Agente, Investigación para la Web}{\textbf{Profesionales}}{Programación, Linux,\\ Desarrollo Web,\\ OpenSource}
\cvcomputer{\textbf{Personales}}{Cine, Web Comics, Música, Leer,\\ Viajar, Tenis}{}{}

\newpage

\section{Experiencia laboral}

\cventry{2007}{Administración servidor}{FUNDACIÓN FORHUM}{Universidad Nacional de Colombia}{Medellín}{
Administrador del servidor AGORA, seguimiento y evaluación de los usos de la plataforma de comunicación y educación virtual, creación de respaldos para los documentos existentes en los diferentes equipos del proyecto de comunicación y educación virtual.\\
Jefe inmediato: Ana Mercedes Múnera Brand.\\
Teléfono: (+57) 4 430 94 27.\\
May 2007 - Jul 2007}

\smallskip

\cventry{2007--2008}{Auxiliar Académico}{Universidad Nacional de Colombia}{Medellín}{}{
Administración del servidor AGORA en la Escuela del Hábitat, realización de respaldos para la información de los cursos virtuales ofertados por la escuela, montaje de contenidos y actualización de la página web. Aporte al proceso de investigación en el diseño del componente de ingeniería de la plataforma de comunicación y educación virtual.\\
Jefe inmediato: Juan Carlos Ceballos.\\
Teléfono: (+57) 4 430 94 29.\\
Ene 2007 $-$ Jun 2008 (3 semestres)}

\smallskip

\cventry{2008--2009}{Desarrollador web}{WeThink Marketing}{Medellin}{}{
Desarrollo de aplicaciones y páginas web para clientes locales y del exterior utilizando tecnologías y herramientas como PHP, Python, JavaScript, HTML y CSS. Configuración de entornos de pruebas y producción en Servidores Privados Virtuales (VPS) bajo Linux. Instalación de directorio LDAP, servicios de correo y otras aplicaciones de uso interno.\\
Jefe inmediato: Carlos Aguirre.\\
Teléfono: (+57) 4 448 03 00.\\
Jul 2008 $-$ Abr 2009}

\smallskip

\cventry{2009}{Estudiante Auxiliar}{Universidad Nacional de Colombia}{Medellín}{}{
Proyecto \emph{ACTUALIZACIÓN DEL CURRICULUM DE INGENIERÍA DE SOFTWARE DE LA ESCUELA DE SISTEMAS}.\\
12 horas semanales}

\smallskip

\cventry{2009--2010}{Becario}{Universidad Nacional de Colombia}{Medellín}{}{
Apoyo como monitor académico en la asignatura Programación Lógica y Funcional.\\
2009-03, 2010-01 (2 semestres)\\
12 horas semanales}

\newpage

\section{Referencias personales}

\smallskip

\cventry{}{Carlos E. Mejía Salar}{Doctor en Matemáticas}{}{}{
Profesor Asociado Escuela de Matemáticas\\
Universidad Nacional de Colombia, Sede Medellín\\
Teléfono: (+57) 4 430 93 23\\ 
\hspace*{1.5cm}(+57) 4 430 93 55}

\smallskip

\cventry{}{Fernando Arango Isaza}{Doctor en Informática}{}{}{
Profesor Asociado Escuela de Sistemas\\
Universidad Nacional de Colombia, Sede Medellín\\
Teléfono: (+57) 4 425 53 30}

\smallskip

\cventry{}{Efrain Rodas}{Contador}{Especialización en Economía y Negocios Internacionales}{}{
Gerente Contraloría Corporativa\\
BANACOL\\
Teléfono: (+57) 320 680 01 04}

\smallskip

\cventry{}{Jorge A. Torres Henao}{Matemático}{}{}{
Jefe de Desarrollo\\
SINAPSIS\\
Teléfono: (+57) 4 338 14 48}

\end{document}


%% end of file `template_en.tex'.
